% page layout
\usepackage{geometry}
\geometry{a4paper, left = 3cm, right = 3cm, top = 3cm, bottom = 3cm}

% fonts and justifying
\usepackage{textcomp}
\usepackage{microtype}
\usepackage{bm}

% mathematics fonts
\usepackage{mathtools}
\usepackage{amsmath} 							
\usepackage{amsfonts}
\usepackage{amssymb}
\setcounter{MaxMatrixCols}{20}

% development tools
\usepackage{soul}

% page layout
\usepackage{fancyhdr}
\setlength{\headheight}{15pt}
\pagestyle{fancy}
\renewcommand{\chaptermark}[1]{ \markboth{#1}{} }
\renewcommand{\sectionmark}[1]{ \markright{#1} }
\fancyhf{}
\fancyhead[LE,RO]{\thepage}
\fancyhead[RE]{\nouppercase{\leftmark}}
\fancyhead[LO]{\nouppercase{\rightmark}}
\fancypagestyle{plain}{
\fancyhf{}
\renewcommand{\headrulewidth}{0pt}
\renewcommand{\footrulewidth}{0pt}}
\rfoot{\footnotesize Probabilistische Datenwissenschaft für die Psychologie
$\vert$ \copyright $ $ 2025 Dirk Ostwald CC BY 4.0}

%  title page
\renewcommand{\maketitle}{
    \begin{titlepage}
    \centering
    
    \vspace*{2cm}
    \includegraphics[width=4cm]{_figures/otto.png}
    
    \vspace{3cm}

    \huge\textbf{Probabilistische Datenwissenschaft} 
    
    \huge\textbf{für die Psychologie}    
    
    \vspace{5cm}
    
    \huge{Dirk Ostwald}

    \vspace{1cm}
    
\end{titlepage}
}

% section titles
\setkomafont{chapter}{\sffamily\bfseries\huge}
\setkomafont{section}{\sffamily\bfseries\Large}

% figure captions
\usepackage{caption}
\captionsetup{
font 		    = footnotesize, 
labelfont       = bf, 
format          = plain,
labelsep 	    = space, 
justification   = justified,
singlelinecheck = false}

% definitions, theorems, proofs
\AtEndEnvironment{definition}{\hfill$\bullet$}
\AtEndEnvironment{theorem}{\hfill$\circ$}
\AtBeginEnvironment{proof}{\footnotesize}

% additional math commands
\usepackage{bm}                                         
\newcommand{\niton}{\not\owns}
\newcommand{\ups}{\mbox{y}}
\newcommand{\tee}{\mbox{t}}
\newcommand{\eps}{\varepsilon}
\DeclareMathOperator*{\intinf}{\int_{-\infty}^{\infty}}
\DeclareMathOperator*{\argmax}{arg\,max}
\DeclareMathOperator*{\argmin}{arg\,min}
\DeclareMathOperator{\tr}{\mbox{tr}}


